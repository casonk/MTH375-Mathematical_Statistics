\documentclass[12pt]{article}

\usepackage{amsthm}
\usepackage{amsmath}
\usepackage{amssymb}
\usepackage{mathtools}
\usepackage{xcolor}
\usepackage{graphicx}
\usepackage{pgfplots}
\usepackage{hyperref}
\usepackage{url}

\usepackage[left = 1cm, top = 2cm, bottom = 3cm, right = 1cm]{geometry}

\newcommand{\XB}{\color{black}}
\newcommand{\XBB}{\color{blue}}
\newcommand{\XV}{\color{violet}}
\newcommand{\XR}{\color{red}}
\newcommand{\ds}{\displaystyle}
\newcommand{\dm}{\displaymath}

\begin{document}

\title{\textbf{MTH375}: Mathematical Statistics - Homework \#7}
\date{\today}
\author{\XV\textit{\large{\href{https://github.com/casonk}{Cason Konzer}}}\XB}

\maketitle
\hrulefill
\vfill 
    \underline{Key Concepts}: Pivotal quantities, confidence intervals.

\newpage
%%%%%%%%%%%%%%%%%%%%%%%%%%%%%%%%%%%%%%%%%%%%%%%%%%%%%%%%%%%%%%%%%%%%%%%%%%%%%%%%%%%%%%%%%%%%%%
%%%%%%%%%     #1     %%%%%%%%%%%%%%%%%%%%%%%%%%%%%%%%%%%%%%%%%%%%%%%%%%%%%%%%%%%%%%%%%%%%%%%%%
%%%%%%%%%%%%%%%%%%%%%%%%%%%%%%%%%%%%%%%%%%%%%%%%%%%%%%%%%%%%%%%%%%%%%%%%%%%%%%%%%%%%%%%%%%%%%%
\newpage
\XBB\hrulefill\XB \\

1. Let $ X_{1}, \dots X_{10} $ be a sample of $ iid \sim Normal(0,\theta) $ random variables. \\ 

\XBB\hrulefill\XB 
\vspace{5mm}

%%%%%%%%%     1a     %%%%%%%%%%%%%%%%%%%%%%%%%%%%%%%%%%%%%%%%%%%%%%%%%%%%%%%%%%%%%%%%%%%%%%%%%
(a) Show that $ \ds T = \frac{1}{\theta} \sum_{i=1}^{n} X_{i}^{2} $ is a pivotal quantity and find its distribution. \\
\vspace{2.5mm} \\
\textit{Solution}:
\vspace{2.5mm} \\

\noindent
A pivotal quantities distribution is does not depend on $ \theta $. \\

\begin{itemize}
    \item $ \ds \frac{1}{\sigma^{2}} \sum_{i=1}^{n} (X_{i} - \overline{X})^{2} \sim \chi_{n-1}^{2}  $. \\
    \item $ \ds f_{T}(t) = \frac{1}{\theta} \sum_{i=1}^{10} (X_{i} - 0)^{2} \sim \chi_{9}^{2}  $. \\
\end{itemize}

\noindent
As the chi squared distribution depends only on the sample size $ n $, $ T $ is a pivotal quantity. \\

\vspace{2.5mm}

%%%%%%%%%     1b     %%%%%%%%%%%%%%%%%%%%%%%%%%%%%%%%%%%%%%%%%%%%%%%%%%%%%%%%%%%%%%%%%%%%%%%%%
(b) Determine a $ 99\% $ confidence interval for $ \theta $ based on $ T $. \\
\vspace{2.5mm} \\
\textit{Solution}:
\vspace{2.5mm} \\

\begin{itemize}
    \item $ \ds \mathbb{P} \Bigl( a \le \frac{1}{\theta} \sum_{i=1}^{10} X_{i}^{2} \le b \Bigr) = 0.99 $. \\
    \item $ \ds \mathbb{P} \Bigl( \frac{1}{b} \sum_{i=1}^{10} X_{i}^{2} \le \theta \le \frac{1}{a} \sum_{i=1}^{10} X_{i}^{2} \Bigr) = 0.99 $. \\
    \item $ \ds {\XBB \texttt{qchisq(p = c(0.005, 0.995), df = 9)}} = (1.735, 23.589) = (a, b) $.
    \item $ \ds L(\overrightarrow{X}) = 0.042 \sum_{i=1}^{10} X_{i}^{2} $. \\
    \item $ \ds U(\overrightarrow{X}) = 0.576 \sum_{i=1}^{10} X_{i}^{2} $. \\
\end{itemize}

\vspace{2.5mm}

%%%%%%%%%     1c     %%%%%%%%%%%%%%%%%%%%%%%%%%%%%%%%%%%%%%%%%%%%%%%%%%%%%%%%%%%%%%%%%%%%%%%%%
(c) Suppose that $ \ds \sum_{i=1}^{10} X_{i}^{2} = 27.3 $. 
What is the $ 99\% $ confidence interval for $ \theta $? \\
\vspace{2.5mm} \\
\textit{Solution}:
\vspace{2.5mm} \\

\begin{itemize}
    \item $ \ds L(\overrightarrow{X}) = 0.042 \cdot 27.3 = 1.157 $. \\
    \item $ \ds U(\overrightarrow{X}) = 0.576 \cdot 27.3 = 15.735 $. \\
\end{itemize}

\noindent
Our $ 99\% $ confidence interval is thus $ (1.157, 15.735) $. \\

\vspace{2.5mm}

%%%%%%%%%     1d     %%%%%%%%%%%%%%%%%%%%%%%%%%%%%%%%%%%%%%%%%%%%%%%%%%%%%%%%%%%%%%%%%%%%%%%%%
(d) Determine a $ 99\% $ confidence interval for $ \text{SD}(X) = \sqrt{\theta} $. \\
\vspace{2.5mm} \\
\textit{Solution}:
\vspace{2.5mm} \\

\noindent
As we know the interval for the variance, the requested is the roots of the above bounds. \\

\begin{itemize}
    \item $ \ds L^{*}(\overrightarrow{X}) = \sqrt{L(\overrightarrow{X})} = \sqrt{1.157} = 1.076 $. \\
    \item $ \ds U^{*}(\overrightarrow{X}) = \sqrt{U(\overrightarrow{X})} = \sqrt{15.735} = 3.967 $. \\
\end{itemize}

\noindent
Our confidence interval is thus $ (1.076, 3.967) $. \\

\vspace{2.5mm}

%%%%%%%%%%%%%%%%%%%%%%%%%%%%%%%%%%%%%%%%%%%%%%%%%%%%%%%%%%%%%%%%%%%%%%%%%%%%%%%%%%%%%%%%%%%%%%
%%%%%%%%%     #2     %%%%%%%%%%%%%%%%%%%%%%%%%%%%%%%%%%%%%%%%%%%%%%%%%%%%%%%%%%%%%%%%%%%%%%%%%
%%%%%%%%%%%%%%%%%%%%%%%%%%%%%%%%%%%%%%%%%%%%%%%%%%%%%%%%%%%%%%%%%%%%%%%%%%%%%%%%%%%%%%%%%%%%%%
\newpage
\XBB\hrulefill\XB \\

2. Let $ X_{1}, \dots X_{6} $ be a sample of $ iid \sim Uniform[\theta, 1] $ random variables. \\ 

\XBB\hrulefill\XB 
\vspace{5mm} 

%%%%%%%%%     2a     %%%%%%%%%%%%%%%%%%%%%%%%%%%%%%%%%%%%%%%%%%%%%%%%%%%%%%%%%%%%%%%%%%%%%%%%%
(a) Show that $ \ds T = \frac{1 - X_{(1)}}{1 - \theta} $ is a pivotal quantity and find its cdf. \\
\vspace{2.5mm} \\
\textit{Solution}:
\vspace{2.5mm} \\

\noindent
A pivotal quantities distribution is does not depend on $ \theta $. \\

\noindent

\begin{itemize}
    \item $ \ds \mathbb{P} \Bigl( X_{(1)} > m \Bigr) = \Bigl[ 1 - F_{X_{i}} \Bigr]^{n} = \Bigl[ 1 - \frac{m - \theta}{1 - \theta} \Bigr]^{n} $. \\
    \item $ \ds \mathbb{P} \Bigl( \frac{1 - X_{(1)}}{1 - \theta} \le t \Bigr) = \mathbb{P} \Bigl( -X_{(1)} \le t(1 - \theta) - 1 \Bigr) = \mathbb{P} \Bigl( X_{(1)} > 1 - t(1 - \theta) \Bigr) = \Bigl[ 1 - \frac{t - \theta}{1 - \theta} \Bigr]^{n} $. \\
    \item Make the substitution $ \ds t = \frac{1 + \theta}{t} + \theta $  
    \item $ \ds \mathbb{P} \Bigl( X_{(1)} > 1 - \Bigl( \frac{1 + \theta}{t} - \theta \Bigr)(1 - \theta) \Bigr) = \Biggl[ 1 - \frac{\ds \frac{1 + \theta}{t} + \theta - \theta}{1 - \theta} \Biggr]^{n} $. \\
    \item $ \ds \mathbb{P} \Bigl( X_{(1)} > 1 - \frac{1 + \theta}{t} \Bigr) = \Bigl[ 1 - \frac{1}{t} \Bigr]^{n} $. \\
\end{itemize}

\noindent
We can see that $ T $ is independent of $ \theta $. \\

\vspace{2.5mm}

\newpage

%%%%%%%%%     2b     %%%%%%%%%%%%%%%%%%%%%%%%%%%%%%%%%%%%%%%%%%%%%%%%%%%%%%%%%%%%%%%%%%%%%%%%%
(b) Determine a $ 85\% $ confidence interval for $ \theta $ based on $ T $. \\
\vspace{2.5mm} \\
\textit{Solution}:
\vspace{2.5mm} \\

\begin{itemize}
    \item $ \ds \mathbb{P} \Bigl( \frac{1 - X_{(1)}}{1 - \theta} \le a \Bigr) = 0.075 = a^{6} \Rightarrow a = 0.075^{1/6} = 0.649 $. \\
    \item $ \ds \mathbb{P} \Bigl( \frac{1 - X_{(1)}}{1 - \theta} \le b \Bigr) = 0.925 = b^{6} \Rightarrow b = 0.925^{1/6} = 0.987 $. \\
    \item $ \ds \mathbb{P} \Bigl( 0.649 \le \frac{1 - X_{(1)}}{1 - \theta} \le 0.987 \Bigr) = 0.85 $. \\
    \item $ \ds \mathbb{P} \Bigl( 1 - \frac{1- X_{(1)}}{0.649} \le \theta \le 1 - \frac{1- X_{(1)}}{0.987} \Bigr) = 0.85 $. \\
    \item $ \ds L(\overrightarrow{X}) = 1 - \frac{1- X_{(1)}}{0.649} $. \\
    \item $ \ds U(\overrightarrow{X}) = 1 - \frac{1- X_{(1)}}{0.987} $. \\
\end{itemize}

\vspace{2.5mm}

%%%%%%%%%     2c     %%%%%%%%%%%%%%%%%%%%%%%%%%%%%%%%%%%%%%%%%%%%%%%%%%%%%%%%%%%%%%%%%%%%%%%%%
(c) Suppose that $ \overrightarrow{X} = \langle 0.527, 0.803, 0.842, 0.880, 0.474, 0.558 \rangle $. 
Find the $ 85\% $ confidence interval for $ \theta $. \\
\vspace{2.5mm} \\
\textit{Solution}:
\vspace{2.5mm} \\

\begin{itemize}
    \item $ \ds X_{(1)} = 0.474 $. \\
    \item $ \ds L(\overrightarrow{X}) = 1 - \frac{1- 0.474}{0.649} = 0.190 $. \\
    \item $ \ds U(\overrightarrow{X}) = 1 - \frac{1- 0.474}{0.987} = 0.467 $. \\
\end{itemize}

\noindent
Our $ 85\% $ confidence interval is thus $ (0.190, 0.467) $. \\

\vspace{2.5mm}

%%%%%%%%%%%%%%%%%%%%%%%%%%%%%%%%%%%%%%%%%%%%%%%%%%%%%%%%%%%%%%%%%%%%%%%%%%%%%%%%%%%%%%%%%%%%%%
%%%%%%%%%     #3     %%%%%%%%%%%%%%%%%%%%%%%%%%%%%%%%%%%%%%%%%%%%%%%%%%%%%%%%%%%%%%%%%%%%%%%%%
%%%%%%%%%%%%%%%%%%%%%%%%%%%%%%%%%%%%%%%%%%%%%%%%%%%%%%%%%%%%%%%%%%%%%%%%%%%%%%%%%%%%%%%%%%%%%%
\newpage
\XBB\hrulefill\XB \\

3. Let $ T = X_{1}, \dots X_{8} $ be a sample of $ iid \sim Gamma(4, \theta) $ random variables. \\

\XBB\hrulefill\XB 
\vspace{5mm} 

%%%%%%%%%     3a     %%%%%%%%%%%%%%%%%%%%%%%%%%%%%%%%%%%%%%%%%%%%%%%%%%%%%%%%%%%%%%%%%%%%%%%%%
(a) Show that $ \ds \frac{2}{\theta} \sum_{i=1}^{n} X_{i} $ is a pivotal quantity and find its distribution. \\
\vspace{2.5mm} \\
\textit{Solution}:
\vspace{2.5mm} \\ 

\noindent
A pivotal quantities distribution is does not depend on $ \theta $. \\

\begin{itemize}
    \item Let $ \ds L = \sum_{i=1}^{8} X_{i} \sim Gamma(32, \theta) $
    \item $ \ds F_{L}(l) = P\Bigl( \frac{2L}{\theta} \le l \Bigr) = F_{L}(l) = P\Bigl( L \le \frac{\theta l}{2} \Bigr) \sim \chi_{64}^{2} $. \\
\end{itemize}

\noindent
As the chi squared distribution depends only on the sample size $ n $, $ L $ is a pivotal quantity. \\

\vspace{2.5mm}

%%%%%%%%%     3b     %%%%%%%%%%%%%%%%%%%%%%%%%%%%%%%%%%%%%%%%%%%%%%%%%%%%%%%%%%%%%%%%%%%%%%%%%
(b) Determine a $ 95\% $ confidence interval for $ \theta $ based on $ T $. \\
\vspace{2.5mm} \\
\textit{Solution}:
\vspace{2.5mm} \\

\begin{itemize}
    \item $ \ds \mathbb{P} \Bigl( a \le \sum_{i=1}^{8} X_{i} \le b \Bigr) = 0.95 $. \\
    \item $ \ds \mathbb{P} \Bigl( \frac{1}{b} \sum_{i=1}^{8} X_{i} \le \theta \le \frac{1}{a} \sum_{i=1}^{8} X_{i} \Bigr) = 0.95 $. \\
    \item $ \ds {\XBB \texttt{qchisq(p = c(0.025, 0.975), df = 64)}} = (43.776, 88.004) = (a, b) $.
    \item $ \ds L(\overrightarrow{X}) = 0.0114 \sum_{i=1}^{8} X_{i} $. \\
    \item $ \ds U(\overrightarrow{X}) = 0.0228 \sum_{i=1}^{8} X_{i} $. \\
\end{itemize}

\vspace{2.5mm}

%%%%%%%%%     3c     %%%%%%%%%%%%%%%%%%%%%%%%%%%%%%%%%%%%%%%%%%%%%%%%%%%%%%%%%%%%%%%%%%%%%%%%%
(c) Suppose that $ \overrightarrow{X} = \langle 17.40, 15.95, 15.17, 7.92, 11.54, 10.46, 8.47, 15.54 \rangle $.
Find a $ 95\% $ confidence interval for $ \theta $. \\
\vspace{2.5mm} \\
\textit{Solution}:
\vspace{2.5mm} \\

\begin{itemize}
    \item $ \ds \sum_{i=1}^{8} X_{i} = 102.45 $
    \item $ \ds L(\overrightarrow{X}) = 0.0114 \cdot 102.45 = 1.164 $. \\
    \item $ \ds U(\overrightarrow{X}) = 0.0228 \cdot 102.45 = 2.340 $. \\
\end{itemize}

\noindent
Our $ 95\% $ confidence interval is thus $ (1.164, 2.340) $. \\

\vspace{2.5mm}

%%%%%%%%%%%%%%%%%%%%%%%%%%%%%%%%%%%%%%%%%%%%%%%%%%%%%%%%%%%%%%%%%%%%%%%%%%%%%%%%%%%%%%%%%%%%%%
%%%%%%%%%     #4     %%%%%%%%%%%%%%%%%%%%%%%%%%%%%%%%%%%%%%%%%%%%%%%%%%%%%%%%%%%%%%%%%%%%%%%%%
%%%%%%%%%%%%%%%%%%%%%%%%%%%%%%%%%%%%%%%%%%%%%%%%%%%%%%%%%%%%%%%%%%%%%%%%%%%%%%%%%%%%%%%%%%%%%%
\newpage
\XBB\hrulefill\XB \\

4. Let $ X_{1}, \dots X_{6} $ be a sample of $ iid \sim Normal(\mu, \theta) $ random variables. \\

\XBB\hrulefill\XB 
\vspace{5mm} 

%%%%%%%%%     4a     %%%%%%%%%%%%%%%%%%%%%%%%%%%%%%%%%%%%%%%%%%%%%%%%%%%%%%%%%%%%%%%%%%%%%%%%%
(a) Why is $ \ds T = \frac{(n-1)S^{2}}{\theta} $ a pivotal quantity? 
What is the distribution of $ \ds \frac{(n-1)S^{2}}{\theta} $? \\
\vspace{2.5mm} \\
\textit{Solution}:
\vspace{2.5mm} \\

\noindent
$ T $ is a pivotal quantity as its distribution does not depend on $ \theta $.

\begin{itemize}
    \item $ \ds \frac{(n-1)S^{2}}{\sigma^{2}} = \frac{1}{\sigma^{2}} \sum_{i=1}^{n} (X_{i} - \overline{X})^{2} \sim \chi_{n-1}^{2}  $. \\
    \item $ \ds f_{T}(t) = \frac{1}{\theta} \sum_{i=1}^{6} (X_{i} - \overline{X})^{2} \sim \chi_{5}^{2}  $. \\
\end{itemize}
\noindent
Conclusion. \\

\vspace{2.5mm}

%%%%%%%%%     4b     %%%%%%%%%%%%%%%%%%%%%%%%%%%%%%%%%%%%%%%%%%%%%%%%%%%%%%%%%%%%%%%%%%%%%%%%% 
(b) Determine a $ 93\% $ confidence interval for $ \theta $ based on $ T $. \\
\vspace{2.5mm} \\
\textit{Solution}:
\vspace{2.5mm} \\ 

\noindent
Intro. \\

\begin{itemize}
    \item $ \ds \mathbb{P} \Bigl( a \le \frac{5S^{2}}{\theta} \le b \Bigr) = 0.93 $. \\
    \item $ \ds \mathbb{P} \Bigl( \frac{5S^{2}}{b} \le \theta \le \frac{5S^{2}}{a} \Bigr) = 0.93 $. \\
    \item $ \ds {\XBB \texttt{qchisq(p = c(0.035, 0.965), df = 5)}} = (0.969, 11.985) = (a, b) $.
    \item $ \ds L(\overrightarrow{X}) = \frac{5S^{2}}{11.985} $. \\
    \item $ \ds U(\overrightarrow{X}) = \frac{5S^{2}}{0.969} $. \\
\end{itemize}

\noindent
Conclusion. \\

\vspace{2.5mm}

%%%%%%%%%     4c     %%%%%%%%%%%%%%%%%%%%%%%%%%%%%%%%%%%%%%%%%%%%%%%%%%%%%%%%%%%%%%%%%%%%%%%%% 
(c) Suppose that $ \overrightarrow{X} = \langle 0.71, 0.62, 2.28, 1.01, 10.01, 7.64 \rangle $.
Find a $ 95\% $ confidence interval for $ \theta $. \\
\vspace{2.5mm} \\
\textit{Solution}:
\vspace{2.5mm} \\ 

\begin{itemize}
    \item $ \ds S(\overrightarrow{X}) = 4.075 $
    \item $ \ds S^{2}(\overrightarrow{X}) = 16.604 $
    \item $ \ds L(\overrightarrow{X}) = \frac{5(16.604)}{11.985} = 6.927 $. \\
    \item $ \ds U(\overrightarrow{X}) = \frac{5(16.604)}{0.969} = 85.643 $. \\
\end{itemize}

\noindent
Our $ 93\% $ confidence interval is thus $ (6.927, 85.643) $. \\

\vspace{2.5mm}

%%%%%%%%%%%%%%%%%%%%%%%%%%%%%%%%%%%%%%%%%%%%%%%%%%%%%%%%%%%%%%%%%%%%%%%%%%%%%%%%%%%%%%%%%%%%%%
%%%%%%%%%     #5     %%%%%%%%%%%%%%%%%%%%%%%%%%%%%%%%%%%%%%%%%%%%%%%%%%%%%%%%%%%%%%%%%%%%%%%%%
%%%%%%%%%%%%%%%%%%%%%%%%%%%%%%%%%%%%%%%%%%%%%%%%%%%%%%%%%%%%%%%%%%%%%%%%%%%%%%%%%%%%%%%%%%%%%%
\newpage
\XBB\hrulefill\XB \\

5. A pollster chooses $ 500 $ citizens at random, and asks whether each plans to vote `yes' or `no' on proposition $ X $.
In the sample, $ 375 $ indicate a `yes' vote. \\

Find a $ 95\% $ confidence interval for the proportion of the total population who plan to vote `yes.'

\XBB\hrulefill\XB 
\vspace{5mm} 

\vspace{2.5mm}
\textit{Solution}:
\vspace{2.5mm} \\

\noindent
We will leveralge the general form of symmetric confidence intervals \dots \\

$ \ds CI = sample \ statistic \pm (critical \ value) (standard \ error) $.  \\

\noindent
Given our sample is sufficently large our sampling distribution is approximately normally distributed, by the central limit theorem. \\

\noindent
From our sample we can construct an estimate for the probablity of a voter panning to vote `yes,' our sample statistic \dots \\ 

$ \hat{p} = 375/500 = 0.75 $. \\

\noindent
For the normal distribution we can solve for values $ (a,b) $, where $ a = -b $ and $ | a | = | b | $ is our critical value \dots \\

$ \ds {\XBB \texttt{qnorm(p = c(0.025, 0.975), mean = 0, sd = 1)}} = (-1.96, 1.96) = (a, b) $. \\

\noindent
From our sample statistic we can construct our standard error \dots \\ 

$ \ds SE = \sqrt{ \ds \frac{\hat{p}(1 - \hat{p})}{n}} = \sqrt{ \ds \frac{(0.75)(0.25)}{500}} = 0.194 $ \\

\noindent
We now have the necessacary information to construct our $ 95\% $ confidence interval \dots \\

CI: $ 0.75 \pm (1.96)(0.194) = 0.75 \pm 0.038 $. \\

\noindent
Our $ 95\% $ confidence interval is thus $ (0.712, 0.788) $.
\end{document}